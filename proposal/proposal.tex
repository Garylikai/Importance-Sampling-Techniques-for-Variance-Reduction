\documentclass[11pt]{article}
\usepackage[margin=1in]{geometry}
\usepackage[utf8]{inputenc}
\usepackage[all]{xy}
\usepackage[hyphens]{url}
\usepackage[hidelinks]{hyperref}

\usepackage{url, hyperref, amsmath, amsfonts, stmaryrd, amssymb, amsthm, mathtools, enumerate, bm, tabularx, graphicx, caption, booktabs, array, ragged2e}
\geometry{letterpaper}

\title{\textbf{AMS 553: Simulation Modeling and Analysis}\\Project Proposal}
\author{Kai Li\\\href{mailto:kai.li@stonybrook.edu}{\texttt{kai.li@stonybrook.edu}} \and Wenbo Du\\\href{mailto:wenbo.du@stonybrook.edu}{\texttt{wenbo.du@stonybrook.edu}} \and Zhe Zhou\\\href{mailto:zhe.zhou@stonybrook.edu}{\texttt{zhe.zhou@stonybrook.edu}} \and Zeyu Dong\\\href{mailto:zeyu.dong@stonybrook.edu}{\texttt{zeyu.dong@stonybrook.edu}}}
\date{Stony Brook University --- \today}

\begin{document}
\maketitle

\section{Project Description}

Importance sampling is an essential variance reduction technique in one of the Monte Carlo methods. The main idea behind importance sampling is that certain values of input random variables in a simulation are more important to the parameter being estimated than others. In some cases, it is very difficult to precisely estimate the probability of a rare event because the estimator's variance is too big. In general, in order to get a meaningful result, we increase the number of replications to reduce the variance of the estimator. Hence, a Crude Monte Carlo simulation generally requires a large number of replications. In this project, we will apply importance sampling techniques in simulation to estimate the probability of a rare event with a much smaller number of replications to achieve a higher statistical efficiency of simulation.

\section{Workload Distribution}
\begin{enumerate}
\item Kai Li will do a thorough research on the existing importance sampling techniques.
\item Wenbo Du will prepare a project presentation.
\item Zhe Zhou will write a proposal and provide relevant code to implement the given method.
\item Zeyu Dong will write a final report of the project.
\end{enumerate}

\end{document} 